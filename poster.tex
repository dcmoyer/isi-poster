% Gemini theme
% https://github.com/anishathalye/gemini

\documentclass[final]{beamer}

% ====================
% Packages
% ====================

\usepackage[T1]{fontenc}
\usepackage{lmodern}
\usepackage[size=custom,width=120,height=72,scale=1.0]{beamerposter}
\usetheme{gemini}
\usecolortheme{usc}

\usepackage{graphicx}
\usepackage{booktabs}
\usepackage{tikz}
\usepackage{pgfplots}
\usepackage{subcaption}

\usepackage{mathtools}
\usepackage{amsmath}

% ====================
% Colors 
% ====================


\definecolor{nice_blue}{RGB}{65, 105, 225}
\definecolor{nice_red}{RGB}{168, 34, 34}
\definecolor{IGCBlue}{HTML}{16197A}

\definecolor{dark_green}{RGB}{20, 110, 10}

\definecolor{graph_blue}{RGB}{144, 195, 212}
\definecolor{graph_purple}{RGB}{195, 144, 212}
\definecolor{graph_green}{RGB}{161, 212, 144}
\definecolor{graph_orred}{RGB}{202, 81, 64}

% ====================
% Lengths
% ====================

% If you have N columns, choose \sepwidth and \colwidth such that
% (N+1)*\sepwidth + N*\colwidth = \paperwidth
\newlength{\sepwidth}
\newlength{\colwidth}
\setlength{\sepwidth}{0.025\paperwidth}
\setlength{\colwidth}{0.3\paperwidth}

\newcommand{\separatorcolumn}{\begin{column}{\sepwidth}\end{column}}

\newcommand{\eat}[1]{}

% ====================
% Title
% ====================

\title{Invariant Representations without Adversarial Training}

%\author{Alyssa P. Hacker \inst{1} \and Ben Bitdiddle \inst{2} \and Lem E. Tweakit \inst{2}}
%\institute[shortinst]{\inst{1} Some Institute \samelineand \inst{2} Another Institute}

\author{Daniel Moyer \and Shuyang Gao \and Rob Brekelmans \and  Greg Ver Steeg \and Aram Galstyan}
\institute[shortinst]{Information Sciences Institute, University of Southern California}

% ====================
% Body
% ====================

\begin{document}

\begin{frame}[t]
\begin{columns}[t]
\separatorcolumn

\begin{column}{\colwidth}

  \begin{alertblock}{The TL;DR}

    \textbf{Have}: Data $x$, Task $\mathcal{L}$, Protected Classes $c$.\\
    \textbf{Want}: Optimal $\mathcal{L}$ performance, unbiased against $c$\\
    \textbf{Idea}: Find a representation $z$ of $x$ invariant to $c$.\\
    %Problem statement:
    \[\min_{f(x)} \mathcal{L}(z,\dots), z=f(x) \text{ s.t. } z \perp c  \]
    Previous works use adversaries. We instead minimize $I(z,c)$.
    This works as well or better \cite{moyer2018invariant}.\\[1em]

    \textbf{Takehome Intution}: Compression $+$ Conditional Reconstruction $\rightarrow$ Invariance \\[1em]

    \begin{center}
      
\begin{tikzpicture}

%Left
%\node[rotate=90] (x) at (-1,0) {$x$ inputs};
\node[rotate=-45] (img) at (-2.0,0.0) {\includegraphics[width=4cm]{images/black-cat-character.png}};
\node (xin) at (-2.0,-2.0) {Input};

\filldraw[fill=gray!33!white, draw=black] (0,-2.25) rectangle (1,2.25);
%\filldraw[fill=gray!33!white, draw=black] (1.25,-3) rectangle (2.25,3);
%\filldraw[fill=gray!33!white, draw=black] (2.5,-2) rectangle (3.5,2);


%Middle
\filldraw[fill=gray!50!white, draw=black,label=z] (5,-1) rectangle (6,1);

%q text
\node (qzx) at (3,0) {Encoder};
\node (z) at (5.5,0) {$z$};

%q lines, top then bot
\draw[thick] (1,-2.25) -- (5,-1);
\draw[thick] (1,2.25) -- (5,1);


%%
%%Here's the offset!
\def\x{8.5}
\def\bump{0.4}

\node[] (brackets) at (9.75,0.0)
  {\Huge$\left\{ \phantom{\sum^{big}_{t}\sum^{big}_{t} }\right\}$};
\node[] (img) at (8 + \bump,0.0) {\includegraphics[width=4cm]{images/black-cat-character.png}};
\draw[ultra thick,black] (9.5 + \bump,-1.25) -- (10.7 + \bump,1.25);

\node[inner sep=1pt] (rot) at (11.5 + \bump,0.0) {\Huge $\circlearrowright$};
\node (zlabel) at (10.0,-2.0) {Invariant Rep.};



%Middle part 2
\filldraw[fill=gray!50!white, draw=black,label=z] (5 + \x,-1) rectangle (6 + \x,1);
\filldraw[fill=blue!33!white, draw=black,label=z] (5 + \x,-2) rectangle (6 + \x,-1);

\node (z) at (5.5 + \x,0) {$z$};
\node (c) at (5.5 + \x,-1.5) {$c$};

%bottom
\filldraw[fill=gray!33!white, draw=black] (10 + \x,-2.25) rectangle (11 + \x,2.25);

%p(x|z) lines, top then bot
\node (pxz) at (8 + \x,-0.5) {Decoder};
\node (pxz) at (8 + \x,0.5) {$\circlearrowright$-Cond.};
\draw[thick] (6 + \x,1) -- (10 + \x,2.25);
\draw[thick] (6 + \x,-2) -- (10 + \x,-2.25);

%x recon branch text
\node[rotate=-45] (img_hat) at (13.5 + \x,0.0) {\includegraphics[width=4cm]{images/black-cat-character.png}};
\node (img_hat) at (13.5 + \x,1.0) {\Huge$\hat{\phantom{XXXX}}$};
\node (xout) at (13.5 + \x,-2.0) {Recon.};

\node (xout) at (0,-3.5) {\footnotesize{Cat Image taken from Yulia Sokolova's illustration tutorial.}};

\end{tikzpicture}



    \end{center}
    \vspace{-0.75cm}

  \end{alertblock}

%%
%%
%%

  \begin{block}{Summary}

  \end{block}

  %%\begin{block}{Pictograms?}
  %%  \begin{minipage}{0.3\textwidth}
  %%    \input{pictogram/cats.tex}
  %%  \end{minipage}
  %%
  %%\end{block}

%%
%%
%%


\end{column}

%%
%%
%%

\separatorcolumn

%%
%%
%%


\begin{column}{\colwidth}

  \begin{block}{Details}
  \end{block}


  \begin{block}{References}

    \footnotesize{
      This work was supported by DARPA grants INPUT NUMBER HERE and INPUT ANOTHER NUMBER HERE, as well as the NSF Graduate Research Fellowship Program Grant Number DGE-1418060. We would like to thank our mothers, our roommates, and the coffee people next door. Also, we'd like to thank Greg for several helpful conversations.
      \bibliographystyle{plain}\bibliography{poster}
    }

  \end{block}

\end{column}

%%
%%
%%

\separatorcolumn

%%
%%
%%

\begin{column}{\colwidth}

  \begin{block}{Demonstration}
  \end{block}
\end{column}

%%
%%
%%

\separatorcolumn
\end{columns}
\end{frame}

\end{document}
